\documentclass[13pt]{article}
\usepackage{tikz}
\usepackage{amssymb}
\usepackage{amsmath}
\newcommand{\Mod}[1]{\ (\text{mod}\ #1)}
\newcommand{\f}[2]{\ensuremath\dfrac{#1}{#2}}
\newcommand{\fp}[2]{\ensuremath\left(\dfrac{#1}{#2}\right)}
\newcommand{\then}{\Rightarrow}
\newcommand{\txt}[1]{\text{ #1 }}
\newcommand{\tand}{\text{ and }}
\newcommand{\tor}{\text{ or }}
\newcommand{\B}[1]{\textbf{ #1 }}
\newcommand{\done}{$\Box$\\\\}
\newcommand{\zz}{\mathbb{Z}}
\newcommand{\D}[2]{#1\ |\ #2}
\newcommand{\N}{\mathbb{N}}
\newcommand{\R}{\mathbb{R}}
\newcommand{\C}{\mathbb{C}}
\newcommand{\zzz}{\mathbb{Z}^+}
\newcommand{\zzd}[1]{\mathbb{Z}[\sqrt{#1}]}
\renewcommand{\gcd}[1]{\text{ gcd}\left(#1\right)}
\newcommand{\be}[1]{
\begin{equation*}
  \begin{split}
    #1
  \end{split}
\end{equation*}
}
\usepackage[margin=1in,top=0.5in,footskip=0.25in]{geometry}
\title{Prime Patterns and Relations}
\author{Cameron Garratt}
\begin{document}
\maketitle
\large
For this paper, I examine the relation of prime numbers to the primes that occur after it. 
I seek to identify interesting patterns in relations. 
This paper is based after the research of Dr. 
Kannan Soundararajan of Stanford University. 
He identified a pattern in the final digit of the prime numbers that I find rather counter intuitive. 
It is well known that primes are very random. 
If you examine the distribution of final digits of prime numbers in base 10, you find that the 1's, 3's, 7's and 9's are distributed relatively evenly regardless of the scale. 
The intuitive assumption would be that primes are random and that there is no influence from one prime to the next. 
Interestingly, Dr Soundararajan found this to not be the case at all. 
Infact, the final digit of a prime number seems to directly influence the final digit of a prime number that occures next.
In this paper, I check the results he finds and attempt to delve deeper into this phenomenon in an attempt to find additional interesting patterns.\\\\
\indent
First we check the results that Dr. Soundararajan finds relating to the final digits of prime distributions. We can see from the following chart that Dr. Soundararajan's results for a depth of 2 in base 10 match my own. Unlike him, I used total percentages instead of percents based on each section of 1's, 3's, 7's and 9's.\\\\
\begin{tabular} { l l l l }
Depth & Count      & Percent   & Pattern\\
\hline
    2 &     446808 &   4.4681\% & 1 $\rightarrow$ 1\\
    2 &     756071 &   7.5607\% & 1 $\rightarrow$ 3\\
    2 &     769923 &   7.6992\% & 1 $\rightarrow$ 7\\
    2 &     526954 &   5.2695\% & 1 $\rightarrow$ 9\\
    2 &     593195 &   5.9320\% & 3 $\rightarrow$ 1\\
    2 &     422302 &   4.2230\% & 3 $\rightarrow$ 3\\
    2 &     714795 &   7.1480\% & 3 $\rightarrow$ 7\\
    2 &     769915 &   7.6992\% & 3 $\rightarrow$ 9\\
    2 &     639384 &   6.3938\% & 7 $\rightarrow$ 1\\
    2 &     681759 &   6.8176\% & 7 $\rightarrow$ 3\\
    2 &     422289 &   4.2229\% & 7 $\rightarrow$ 7\\
    2 &     756851 &   7.5685\% & 7 $\rightarrow$ 9\\
    2 &     820368 &   8.2037\% & 9 $\rightarrow$ 1 MAX \\  
    2 &     640076 &   6.4008\% & 9 $\rightarrow$ 3\\
    2 &     593275 &   5.9328\% & 9 $\rightarrow$ 7\\
    2 &     446032 &   4.4603\% & 9 $\rightarrow$ 9\\
\end{tabular}\\\\
What I find interesting about this is that the final digits of the primes in base 10 are very evenly distributed. The intuitive idea would be that each prime would be independent of any of the others when you get to large numbers. This finding by Dr. Soundararajan disproves this and shows a very counter intuitive notion that each prime influences the prime that comes after. The final digit of the primes tend to oscillate. This pattern is interesting and made me wonder if there are possilby patterns of final digits that occur with a larger frequency than others.
\\\\\indent
Next we look at further patters to see if there are some patterns the the primes that will always be more common than others.
The code provided can analyze the first 2 billion primes up to a specified depth and provide the most common final digit patters that occur.
When analyzing the first 10 million primes in base 10, the max patterns at different depths are:\\\\
\begin{tabular}{ l l l r }
  Depth & Number & Percent & Pattern \\
  \hline
  1 & 2500283 & 25.0028\% & 7\\
  2 & 820368 & 8.2037\% & 9 $\rightarrow$ 1\\
  3 & 256932 & 2.5693\% & 9 $\rightarrow$ 1 $\rightarrow$ 7\\
  4 & 81324 & 0.8132\% & 3 $\rightarrow$ 9 $\rightarrow$ 1 $\rightarrow$ 7\\
  5 & 24413 & 0.2441\% & 1 $\rightarrow$ 3 $\rightarrow$ 9 $\rightarrow$ 1 $\rightarrow$ 7\\
  6 & 7872 & 0.0787\% & 9 $\rightarrow$ 1 $\rightarrow$ 3 $\rightarrow$ 9 $\rightarrow$ 1 $\rightarrow$ 7\\
  7 & 2521 & 0.0252\% & 3 $\rightarrow$ 9 $\rightarrow$ 1 $\rightarrow$ 3 $\rightarrow$ 9 $\rightarrow$ 1 $\rightarrow$ 7\\
  8 & 758 & 0.0076\% & 7 $\rightarrow$ 3 $\rightarrow$ 9 $\rightarrow$ 1 $\rightarrow$ 3 $\rightarrow$ 9 $\rightarrow$ 1 $\rightarrow$ 7\\
  \end{tabular}
\\\\The 10 most common patterns at depth 5 over a million primes are\\\\
First 10 million\\\\
\begin{tabular}{ l l l r }
  Depth & Number & Percent & Pattern \\
  \hline
    5 & 24413 & 0.2441\% & 1 $\rightarrow$ 3 $\rightarrow$ 9 $\rightarrow$ 1 $\rightarrow$ 7\\
    5 & 24270 & 0.2427\% & 9 $\rightarrow$ 1 $\rightarrow$ 3 $\rightarrow$ 9 $\rightarrow$ 1\\
    5 & 24183 & 0.2418\% & 9 $\rightarrow$ 1 $\rightarrow$ 7 $\rightarrow$ 9 $\rightarrow$ 1\\
    5 & 24167 & 0.2417\% & 3 $\rightarrow$ 9 $\rightarrow$ 1 $\rightarrow$ 7 $\rightarrow$ 9\\
    5 & 23069 & 0.2307\% & 3 $\rightarrow$ 9 $\rightarrow$ 1 $\rightarrow$ 3 $\rightarrow$ 9\\
    5 & 22961 & 0.2296\% & 7 $\rightarrow$ 9 $\rightarrow$ 1 $\rightarrow$ 3 $\rightarrow$ 9\\
    5 & 22904 & 0.2290\% & 1 $\rightarrow$ 7 $\rightarrow$ 9 $\rightarrow$ 1 $\rightarrow$ 3\\
    5 & 22866 & 0.2287\% & 1 $\rightarrow$ 7 $\rightarrow$ 3 $\rightarrow$ 9 $\rightarrow$ 1\\
    5 & 22821 & 0.2282\% & 9 $\rightarrow$ 1 $\rightarrow$ 7 $\rightarrow$ 3 $\rightarrow$ 9\\
    5 & 22699 & 0.2270\% & 1 $\rightarrow$ 7 $\rightarrow$ 9 $\rightarrow$ 1 $\rightarrow$ 7\\
  \end{tabular}
\\\\\\First 100 million\\\\
\begin{tabular}{ l l l r }
  Depth & Number & Percent & Pattern \\
  \hline
  5 & 220752 & 0.2208\% & 9 $\rightarrow$ 1 $\rightarrow$ 3 $\rightarrow$ 9 $\rightarrow$ 1\\
  5 & 220393 & 0.2204\% & 1 $\rightarrow$ 3 $\rightarrow$ 9 $\rightarrow$ 1 $\rightarrow$ 7\\
  5 & 220347 & 0.2203\% & 9 $\rightarrow$ 1 $\rightarrow$ 7 $\rightarrow$ 9 $\rightarrow$ 1\\
  5 & 218761 & 0.2188\% & 3 $\rightarrow$ 9 $\rightarrow$ 1 $\rightarrow$ 7 $\rightarrow$ 9\\
  5 & 210841 & 0.2108\% & 1 $\rightarrow$ 7 $\rightarrow$ 9 $\rightarrow$ 1 $\rightarrow$ 3\\
  5 & 210425 & 0.2104\% & 7 $\rightarrow$ 9 $\rightarrow$ 1 $\rightarrow$ 3 $\rightarrow$ 9\\
  5 & 209231 & 0.2092\% & 1 $\rightarrow$ 7 $\rightarrow$ 3 $\rightarrow$ 9 $\rightarrow$ 1\\
  5 & 208463 & 0.2085\% & 3 $\rightarrow$ 9 $\rightarrow$ 1 $\rightarrow$ 3 $\rightarrow$ 9\\
  5 & 208337 & 0.2083\% & 9 $\rightarrow$ 1 $\rightarrow$ 7 $\rightarrow$ 3 $\rightarrow$ 9\\
  5 & 207712 & 0.2077\% & 1 $\rightarrow$ 7 $\rightarrow$ 9 $\rightarrow$ 1 $\rightarrow$ 7\\
\end{tabular}
 \\\\\\It is interesting to note that the most common patterns change slightly when analyzing more primes. However the changes are minimal and the same general patterns are still very common regardless of the number of primes analyzed.
\\\\\\\\1 Billion primes with depth 7\\\\
\begin{tabular}{ l l l r }
  Depth & Number & Percent & Pattern \\
  \hline
  7 & 187110 & 0.0187\% & 7 $\rightarrow$ 9 $\rightarrow$ 1 $\rightarrow$ 3 $\rightarrow$ 9 $\rightarrow$ 1 $\rightarrow$ 7\\
  7 & 185850 & 0.0186\% & 3 $\rightarrow$ 9 $\rightarrow$ 1 $\rightarrow$ 7 $\rightarrow$ 9 $\rightarrow$ 1 $\rightarrow$ 3\\
  7 & 184150 & 0.0184\% & 3 $\rightarrow$ 9 $\rightarrow$ 1 $\rightarrow$ 7 $\rightarrow$ 9 $\rightarrow$ 1 $\rightarrow$ 7\\
  7 & 184041 & 0.0184\% & 3 $\rightarrow$ 9 $\rightarrow$ 1 $\rightarrow$ 3 $\rightarrow$ 9 $\rightarrow$ 1 $\rightarrow$ 7\\
  7 & 182959 & 0.0183\% & 7 $\rightarrow$ 9 $\rightarrow$ 1 $\rightarrow$ 7 $\rightarrow$ 9 $\rightarrow$ 1 $\rightarrow$ 3\\
  7 & 182869 & 0.0183\% & 7 $\rightarrow$ 9 $\rightarrow$ 1 $\rightarrow$ 3 $\rightarrow$ 9 $\rightarrow$ 1 $\rightarrow$ 3\\
  7 & 178559 & 0.0179\% & 9 $\rightarrow$ 1 $\rightarrow$ 7 $\rightarrow$ 3 $\rightarrow$ 9 $\rightarrow$ 1 $\rightarrow$ 7\\
  7 & 178017 & 0.0178\% & 3 $\rightarrow$ 9 $\rightarrow$ 1 $\rightarrow$ 7 $\rightarrow$ 3 $\rightarrow$ 9 $\rightarrow$ 1\\
  7 & 177663 & 0.0178\% & 3 $\rightarrow$ 7 $\rightarrow$ 9 $\rightarrow$ 1 $\rightarrow$ 3 $\rightarrow$ 9 $\rightarrow$ 1\\
  7 & 177647 & 0.0178\% & 1 $\rightarrow$ 7 $\rightarrow$ 9 $\rightarrow$ 1 $\rightarrow$ 3 $\rightarrow$ 9 $\rightarrow$ 1\\
  \end{tabular}\\\\
  These results are interesting because it shows that there are infact some patterns that occur with much more regularity than others. The most common patterns are all very similar. They all oscillate as much as possilbe at every digit. What interests me is why some patterns that appear almost identical show up with a larger frequency. \\\\
  Here is a list of the most common patterns at each depth up to 9 when analyzing the first 500 million primes. \\\\
\begin{tabular} { l l l r }
Depth & Count      & Percent   & Pattern\\
\hline
    1 &  125002060 &  25.0004\% & 7\\
    2 &   39369475 &   7.8739\% & 9 $\rightarrow$ 1\\
    3 &   11785061 &   2.3570\% & 3 $\rightarrow$ 9 $\rightarrow$ 1\\
    4 &    3562197 &   0.7124\% & 3 $\rightarrow$ 9 $\rightarrow$ 1 $\rightarrow$ 7\\
    5 &    1045672 &   0.2091\% & 9 $\rightarrow$ 1 $\rightarrow$ 3 $\rightarrow$ 9 $\rightarrow$ 1\\
    6 &     321442 &   0.0643\% & 9 $\rightarrow$ 1 $\rightarrow$ 3 $\rightarrow$ 9 $\rightarrow$ 1 $\rightarrow$ 7\\
    7 &      96653 &   0.0193\% & 7 $\rightarrow$ 9 $\rightarrow$ 1 $\rightarrow$ 3 $\rightarrow$ 9 $\rightarrow$ 1 $\rightarrow$ 7\\
    8 &      28313 &   0.0057\% & 9 $\rightarrow$ 1 $\rightarrow$ 7 $\rightarrow$ 9 $\rightarrow$ 1 $\rightarrow$ 3 $\rightarrow$ 9 $\rightarrow$ 1\\
    9 &       8649 &   0.0017\% & 3 $\rightarrow$ 9 $\rightarrow$ 1 $\rightarrow$ 7 $\rightarrow$ 9 $\rightarrow$ 1 $\rightarrow$ 3 $\rightarrow$ 9 $\rightarrow$ 1\\
\end{tabular}\\\\
I find the difference between depth 7 and 8 interesting. It looks like most of the patterns at lower depths are contained in the patterns at higher depths. However, this pattern breaks for depth 7 and 8. This makes me wonder if the pattern shown at depth 7 would be contained in a larger depth than 9. Unfortunately, analyzing larger depths takes an exponentially larger amount of time (especially with unoptimized code) so it would take a large amount of time to compute more that a depth of 9 for the first 500 million primes. Seeing these patterns made me wonder about other bases. Sadly, I realized too late that the main datastructure I was using to store information was taking advantage of primes in base 10 to perform optimizations. To change this so that it would work for any base would require a rewrite the main datastructure and consequently a rewrite of nearly all of the code. This is not possible with current time constraints. \\\\
The code used in this paper will be submitted with the project and can also be found at: \\\\https://github.com/cag51574/PrimePatterns\\
\\Notice: you must have java 8 installed on your system to run this project as well as having downloaded and extracted the first 2 billion primes into a directory called primes.
The first 2 billion primes can be downloaded from:\\\\
http://www.primos.mat.br/2T\_en.html\\\\
Please follow the instructions in the README to run the code.
There is a built in user interface that will allow you to perform your own analysis for any depth for the first two billion primes.
Larger depths will take an exponentially larger amount of time, so I recommend staying below a depth of 10.\\\\

\end{document}      
%%% Local Variables:
%%% mode: latex
%%% TeX-master: t
%%% End:







