\documentclass[13pt]{article}
\usepackage{tikz}
\usepackage{amssymb}
\usepackage{amsmath}
\newcommand{\Mod}[1]{\ (\text{mod}\ #1)}
\newcommand{\f}[2]{\ensuremath\dfrac{#1}{#2}}
\newcommand{\fp}[2]{\ensuremath\left(\dfrac{#1}{#2}\right)}
\newcommand{\then}{\Rightarrow}
\newcommand{\txt}[1]{\text{ #1 }}
\newcommand{\tand}{\text{ and }}
\newcommand{\tor}{\text{ or }}
\newcommand{\B}[1]{\textbf{ #1 }}
\newcommand{\done}{$\Box$\\\\}
\newcommand{\zz}{\mathbb{Z}}
\newcommand{\D}[2]{#1\ |\ #2}
\newcommand{\N}{\mathbb{N}}
\newcommand{\R}{\mathbb{R}}
\newcommand{\C}{\mathbb{C}}
\newcommand{\zzz}{\mathbb{Z}^+}
\newcommand{\zzd}[1]{\mathbb{Z}[\sqrt{#1}]}
\renewcommand{\gcd}[1]{\text{ gcd}\left(#1\right)}
\newcommand{\be}[1]{
\begin{equation*}
  \begin{split}
    #1
  \end{split}
\end{equation*}
}
\usepackage[margin=1in,top=0.5in,footskip=0.25in]{geometry}
\title{Prime Patterns and Relations}
\author{Cameron Garratt}
\begin{document}
\maketitle
\large
(Move this to end!!!!!!!!) The code used in this paper will be submitted with the project and can also be found at: https://github.com/cag51574/PrimePatterns\\ Notice that you must have java 8 to run this project.\\\\
For this paper, I examine the relation of prime numbers to the primes that occur after it. 
I seek to identify interesting patterns in relations. 
This paper is based after the research of Dr. 
Kannan Soundararajan of Stanford University. 
He identified a pattern in the final digit of the prime numbers that I find rather counter intuitive. 
It is well known that primes are very random. 
If you examine the distribution of final digits of prime numbers in base 10, you find that the 1's, 3's, 7's and 9's are distributed relatively evenly regardless of the scale. 
The intuitive assumption would be that primes are random and that there is no influence from one prime to the next. 
Interestingly, Dr Soundararajan found this to not be the case at all. 
Infact, the final digit of a prime number seems to directly influence the final digit of a prime number that occures next.
In this paper, I check the results he finds and attempt to delve deeper into this phenomenon in an attempt to find additional interesting patterns.\\\\
\indent
First we check the results that Dr. Soundararajan finds relating to the final digits of prime distributions. 



 



\end{document}      
%%% Local Variables:
%%% mode: latex
%%% TeX-master: t
%%% End:







